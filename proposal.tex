\documentclass[12pt,letterpaper]{article}
\usepackage{acl2010}

\title{Literature Review (cs424)}
\author{Gabor Angeli}

\begin{document}

\maketitle

\section{Task Definition}
For this literature review, papers were chosen in the area of automatically predicting review helpfulness.
The task is generally applied to domains where there are annotated training examples in which
	reviewers can not only rate a product/service, but also rate the helpfulness of another
	reviewer's comment.
The two primary such domains covered were Amazon reviews (particularly cameras and mp3 players) 
	and IMDB movie reviews, among a few others.

The motivation for this task is many-fold.
One motivation given in the literature is to predict helpfulness of reviews which do not yet have a well-established
	human-annotated rating.
This is often the case for new reviews, which never receive user attention, since they are not displayed
	near the top of the page by merit of either being an early review, or being a well-liked review.
On a similar note, reviews which are thoroughly reviewed tend to garner more attention than those which
	are not, and therefore the established reviews tend to receive the most additional helpfulness ratings,
	independently of whether they are in fact the best reviews.

Another motivation is for text summarization.
In this case, reviews which are deemed to be the most helpful should receive a higher weighting in the
	summarization task. 
Current summarization systems tend to consider all reviews as equally informative a-priori.

A last motivation, given in \newcite{2007ghose-helpfulness}, is to assist corporations marketing their
	products assess which aspects of the product the public likes and dislikes.
The assumption is that higher rated reviews express more prominent opinions, both positive and negative.

Five articles are included in this literature review. 
Three of them deal directly with assessing the
	helpfulness of user reviews (\newcite{2006kim-helpfulness}, \newcite{2007liu-helpfulness}, 
	\newcite{2008liu-helpfulness}).
Another approaches the problem with a business-oriented motivation \cite{2007ghose-helpfulness}.
The last deals with sentiment summarization, and is included as an in-depth exploration
	of one of the applications of helpfulness prediction \cite{2008titov-summarization}.


The articles included in the literature review are:

\begin{itemize}
	\item \newcite{2006kim-helpfulness}: {\em Automatically Assessing Review Helpfulness.}
	\item \newcite{2007liu-helpfulness}: {\em Low-Quality Product Review Detection in Opinion Summarization.}
	\item \newcite{2008liu-helpfulness}: {\em Modeling and Predicting the Helpfulness of Online Reviews.}
	\item \newcite{2007ghose-helpfulness}: {\em Designing novel review ranking systems: predicting the usefulness and impact of reviews.}
	\item \newcite{2008titov-summarization}: {\em A Joint Model of Text and Aspect Ratings for Sentiment Summarization.}
\end{itemize}

\section{Summaries of Articles}
\subsection{Automatically Assessing Review Helpfulness \cite{2006kim-helpfulness}}
This paper was the earliest work present in the literature on the subject of assessing review helpfulness.
The main idea of the paper was to use SVN regression on a large number of features,
	analyzing the results and the impact of each subset of features on the performance
	of the system.
The paper concluded that a relatively simple feature set (length of the review,
	unigram counts, and star rating) performed the best, with the first two
	(length and unigram counts) alone being almost as effective as all three
	combined.

The dataset used by the paper was a corpus of Amazon reviews, in the categories
	of {\em MP3 Players} and {\em Digital Cameras}.
The dataset consisted of 821 products and 33,016 reviews for {\em MP3 Players},
	and 1,104 products and 26,189 reviews for {\em Digital Cameras}.
This data was then filtered for duplicate or near-duplicate entries ($>80\%$ bigrams match),
	resulting in 85 products and 12,07 reviews being discarded for {\em MP3 Players},
	and 38 products and 3,692 reviews being discarded for {\em Digital Cameras}.
Helpfulness ratings were obtained by taking every review with more than 5 helpfulness
	responses.
This resulted in approximately a third to half of the reviews being discarded.

Training was done using SVM regression.
The authors tested on a variety of kernels, however found a radial bias function (RBF)
	kernel to perform the best; all results are reported using this method.

To assist in feature creation, a {\em Product-Feature} set was automatically extracted.
This was done by mining references to product features from {\tt Epinions.com},
	where users are allowed to describe the pros and cons of each product.
Frequent words were pruned from this list, resulting in around 10,000 unique features
	for both domains.

Features were extracted over each review.
These features fell into the following categories:
\begin{enumerate}
	\item {\bf Structural}: review length ({\tt LEN}); average sentence length, 
		number of sentences, etc. ({\tt SEN}); 
		HTML formatting ({\tt HTM})
	\item {\bf Lexical}: {\it tf-idf} statistic on each unigram ({\tt UGR});
		{\it tf-idf} on each bigram ({\tt BGR})
	\item {\bf Syntactic} ({\tt SYN}): features on POS information
	\item {\bf Semantic}: product features ({\tt PRF}); 
		{\it General Inquirer} sentiment words ({\tt GIW}) 
	\item {\bf Meta Info}: star rating (average/deviation) ({\tt STR})
\end{enumerate}

Evaluation was done using the Spearman correlation coefficient.
The best results came from the three features ({\tt LEN+UGR+STR}),
	resulting in a Spearman coefficient of 0.656 on {\em MP3 Players}
	and 0.595 on {\em Digital Cameras}.
Adding additional features tended to hurt performance; adding every feature
	dropped the {\em MP3 Player} score to 0.601,
	although mildly improving the {\em Digital Camera} score to 0.604.

\subsection{Low-Quality Product Review Detection in Opinion Summarization \cite{2007liu-helpfulness}}

This paper, like the previous, deals with identifying helpfulness in product reviews. 
Unlike the previous; however, it is concerned with identifying {\em bad} reviews rather
	than establishing a ranking of review helpfulness.
Furthermore, the paper approaches the task in the context of improving summarization by excluding
	poor quality reviews.

The general approach of the paper is to re-define the qualities of a good review based on their
	own criteria, which is shown to correlate with the corpus data.
Then, an SVM is trained from this cleaner corpus, and a summarization system is run on the reviews
	which are deemed to be good quality.
Evaluation is done both on the helpfulness-detection and opinion-summarization aspects of the task.
The paper uses the Amazon corpus of \newcite{2006kim-helpfulness}, taking only the {\em Digital Cameras}
	portion. 

As motivation for re-annotating reviews, the paper notes three types of bias in the Amazon corpus:
\begin{itemize}
	\item {\bf Imbalance vote bias}: users tend to value others' opinions positively.
		Over half of the corpus has $>90\%$ {\em helpful} votes.
	\item {\bf Winner circle bias}: the more votes a review gets, the more authority it appears to have.
	\item {\bf Early bird bias}: the earlier a review is posted, the more votes it tends to get.
\end{itemize}

To adjust for this bias, reviews are re-ranked based by two human annotators based on a custom
	specification for quality.
Reviews are ranked as one of {\em best}, {\em good}, {\em fair}, and {\em bad}.
{\em Best} reviews encompass many aspects of the review, with well-supported evidence and clear writing.
{\em Bad} reviews are characterized by incorrect or misleading information, or off-topic rambling.
Reviews which are {\em Bad} are considered low-quality; every other category is acceptable.

The two hand-annotators were found to agree on this custom annotation often (the paper
	reports a {\em kappa} statistic of 0.8142), however often disagree the
	Amazon ground truth (the hand annotators would switch the order of two
	Amazon reviews on their helpfulness rating nearly 45\% of the time).
The authors justify that their ranking aligns better with a third annotator's judgement than
	the Amazon ratings do.

Training was done with a standard SVM classifier.
The paper describes a large number of features used; the set is summarized roughly below:
\begin{itemize}
	\item {\bf Features on informativeness}: sentence statistics; word statistics; phrase features; product-features features.
	\item {\bf Features on readability}: paragraph statistics; paragraph separator statistics.
	\item {\bf Features on subjectiveness}: sentence sentiment statistics.
\end{itemize}

Evaluation was done by splitting the annotated set of the first annotator into two halves;
	the first half was used as training data, and evaluation was done on both the second
	half of the data and the entire corpus produced by the second annotator.
The accuracy obtained on the first annotator's test set was 83.93\% (using all but
	the {\em subjectiveness} features, while the accuracy
	on the second annotator's complete corpus was 82.96\% (using all the features).
Results using only the informativeness features would yield accuracies of 83.30\% and 82.37\%.

The paper demonstrates that removing low quality reviews appears to improve the summarization task.
It shows that less text fragments are extracted, and that the resulting summaries are significantly different.
It shows that the summaries obtained by this filtering approach are closer to the expert reviews
	collected from CNET.

\subsection{Modeling and Predicting the Helpfulness of Online Reviews \cite{2008liu-helpfulness}}

The approach taken in this paper is similar to the previous two (although note that the author
	has no relation to \newcite{2007liu-helpfulness}); however, the authors focus on different features
	and emphasize more strongly the training mechanism of the system.
The paper tests on IMDB reviews.

In addition to the insights into the task mentioned by either of the previous two papers (notably the early
	bird and winner's circle biases), the authors note that spam votes are a non-negligible source of bias.

The paper's approach is to characterize helpfulness on three factors:
\begin{itemize}
	\item {\bf Reviewer expertise}: users who frequently review a genre tend to be more helpful.
	\item {\bf Writing style}: readable reviews tend to be more helpful.
	\item {\bf Timeliness of review}: reviews posted early tend to be more helpful.
\end{itemize}

The dataset used in the paper was created by the authors by scraping the IMDB website.
A total of 504 movies spanning 27 genres were collected between January 2006 and November 2007,
	resulting in 94,919 reviews from 56,588 reviewers

Training was done using an SVM classifier using a radial bias function (similar to \newcite{2006kim-helpfulness}).
Each of the three main factors mentioned above is incorporated into the SVM. 
Reviewer expertise was incorporated by measuring the similarity of the movie in question (represented as a vector
	of its genres) to the movies previously reviewed by the reviewer.
Reviewers who have reviewed less than a given threshold of movies were ignored for this aspect.
Writing style was incorporated by taking statistics on shallow POS features (the paper claims lexical features
	do not provide predictive power).
Timeliness was incorporated by modeling the drop in review helpfulness as an exponential decay.

The model evaluates both on deviation (similar to \newcite{2006kim-helpfulness}) and retreival (similar
	to \newcite{2007liu-helpfulness}).
The metrics used for these two tasks are {\em Mean Square Error} (MSE) and F$_1$ scores respectively.

Results are given for varying a number of parameters in training; the best result reported is
	an MSE score of 0.0332 and an F$_1$ score of 0.7116.
This was better than the scores for taking any of the three factors independently, although no results
	are given for any pairs of factors.

\subsection{Designing Novel Review Ranking Systems: 
	Predicting the Usefulness and Impact of Reviews \cite{2007ghose-helpfulness}}

The paper's primary goal is to create a helpfulness analysis system 
	for businesses to aid in in adapting to customer reviews.
The paper uses its own dataset of reviews scraped from Amazon covering
	a wider range of categories, though remaining in the realm of consumer electronics.
The general strategy outlined in the paper is to first separate objective and subjective
	sentences, and measure the effects of the subjective sentences on helpfulness.

The dataset used in the paper consists of Amazon reviews from a number of consumer electronics
	categories, although only the audio/video players and digital cameras were used.
The data was collected between March 2005 and May 2006, and incorporates both review information
	(text, stars, helpfulness), and product and sales data (price, sales rank, release date,
	percentage sold as used).
From statistics given in the paper, it appears there were 18,720 examples in total.

The reviews are partitioned into sentences which are objective and those which are subjective.
The paper opts for an approach they claim is similar in spirit to \cite{2004pang-mincuts}, training on
	each type of review separately, and opting for a Dynamic Language Model classifier.
Statistics (average, std. dev.) on 
	the subjectivity of the entire review were compiled based on this per-sentence subjectivity
	classification.

Prediction of both sales rank and helpfulness were reported in the paper. 
For the purposes of this summary, only the helpfulness aspect is reported on.

Training was presented very briefly in the paper.
It appears training was done using a linear classifier, using features including: 
The log of the time since the product was published (a similar result to \cite{2008liu-helpfulness}
	with their exponential decay);
the log of the length of the review;
and, the mean (fraction of) and standard deviation of subjective sentences.

The paper reports F$_1$ scores for both the {\em Audio/Video} and {\em Digital Camera} categories as 0.85.


\subsection{A Joint Model of Text and Aspect Ratings for Sentiment Summarization \cite{2008titov-summarization}}

This paper presents an approach to summarization, jointly learning the topics to summarize and text which
	describes them.
The paper proposes a model ({\em Multi-Aspect Sentiment Model}) consisting of two parts: 
	an unsupervised topic model, and a classifier from words to sentiment ratings.
The paper evaluates on a hotel review dataset taken from {\tt TripAdvisor.com}.

The paper presents the task of summarization as a two-fold task: the first task
	is described as {\em aspect identification and mention extraction} -- determining which aspects of the
	reviews are relevant to describe, and determining which text fragments describe them.
The second task is {\em sentiment classification} -- determining the sentiment on the
	relevant extracted text.
The paper attempts to incorporate both of these tasks into a single model which extracts text fragments
	and their associated rating, given the review and a per-aspect rating.

The dataset used in the paper consists of 10,000 reviews from {\tt TripAdvisor.com}, where each
review was rated in at least {\em service}, {\em location}, and {\em rooms}.

The approach taken is the build a model -- coined as a Multi-Aspect Sentiment model (MAS) --
	which is effectively built on a combination of a multi-grain LDA topic model and a series of MaxEnt
	classifiers for each topic.
The model is such that a word in the document is sampled from either a local or global topic;
	the intent is that global topics will capture topics corresponding to non-sentiment phenomena
	(e.g. {\em MP3 players} versus {\em hotels}), while the local topics will capture sentiment-laden
	words.

The paper raises the issue that often aspects of reviews correlate strongly with each other.
That is, if you dislike a hotel, you will likely not rate any aspect highly.
To address this, the model classifies not over absolute ratings, but over the difference between
	the aspect rating and the overall rating.

Inference over the model was done using Gibbs sampling, as exact inference is intractable.
The model achieves a precision of between 75\% and 85\% for the different aspects.


\section{Comparison of Articles}

\subsection{Comparison of Approaches}
The papers presented in this review cover three tasks: helpfulness prediction, bad review detection,
	and sentiment summarization. 

The standard approach in more or less each of the helpfulness prediction and bad review detection
	tasks (3 of the 4) was to train an SVM classifier on a set of features on the review.
Furthermore, two of the papers were in agreement that radial bias functions provided the best kernel.

\subsection{Comparison of Features}
The characteristic difference between most of the systems is the features they use; and, by extension,
	the phenomena they attempt to capture.

Commonly used features included sentence and word level statistics 
	\cite{2007liu-helpfulness} \cite{2006kim-helpfulness};
	metadata such as stars or user information \cite{2008liu-helpfulness} \cite{2006kim-helpfulness};
	sentiment-laden words \cite{2007liu-helpfulness} \cite{2008liu-helpfulness};
	product features;
	and time-sensitivity.

The union of all the phenomena that these features attempt to capture, in turn, is enumerated below:
\begin{itemize}
	\item Reviewer expertise
	\item Writing style
	\item Timeliness
	\item Imbalanced votes
	\item {\em Winner's circle} phenomenon
\end{itemize}

\subsection{Comparison of Results}
A notably discouraging theme among the papers is a lack of cross-referencing of results.
In fact, nearly every paper has its own dataset, even when working in the same domain
	as previous work.

None the less, a comparison can be made on the effectiveness of various features
	on the different domains.
For example, while \newcite{2006kim-helpfulness} found length to be a valuable feature, the
	results of \newcite{2008liu-helpfulness} reported the opposite finding.
In a related vein, \newcite{2006kim-helpfulness} found that features on sentiment-laden
	words was not significant, whereas \newcite{2007liu-helpfulness} opted for sentiment
	features over shalow syntactic features.

Furthermore, although the testing methodology of \newcite{2007ghose-helpfulness} is not entirely
	well documented, their dataset can be considered similar to that of \newcite{2007liu-helpfulness}.
Comparing the two papers, we see that the first reports an F$_1$ score of 85, while the second reports
	accuracies of 83.93\% and 82.96\%.



\section{Future Work}
The work in the literature seems to have more or less exhaustively explored the area of helpfulness recognition
	with simple features.
Features which were not tried, or discounted without argument (e.g. features on parse trees, 
	vocabulary sophistication, sentiment words expressing {\em modest} versus {\em bold}, etc.) seem unlikely
	to have a large impact on performance.

However, there do seem to be several less trivial extensions to the topic:

\begin{itemize}
	\item {\bf Time-aware features}: each of the current approaches handles time as an independent phenomenon.
		Namely, as time progresses, review helpfulness goes down.
		It seems more intuitive, however, to model the $n$th review's helpfulness in terms of the previous $n-1$
			reviews, rather than as an independent review given a certain time after the product's posting.
		In this way, it might be possible to differentiate posts repeat previous points (unhelpful)
			and those which introduce new content (helpful).
	\item {\bf Topic-aware features}: It seems useful to have a notion of exactly
			what topics a post discusses, and whether these are valuable; or, combined with the above point, whether
			they are novel.
		This is similar to some extent to the product-features features, but should optimally be more coarse-grained.
		For instance, describing the focal length of a camera and its lens are far closer in topic than, say, battery
			life and its lens.
	\item {\bf Deeper summarization integration}: Although one paper described using helpfulness to assist in
			summarization, it seems as if the integration could be made tighter.
		The two tasks seem inherently similar: identifying useful reviews or parts of reviews.
		Particularly if point 2 is feasible, it does not seem like a far jump to pick the most helpful text segments
			(based on other features) from each topic to provide an aggregate review.
\end{itemize}

These extensions can be roughly summarized as an attempt to capture the underlying process which contributes to
	the helpful/unhelpful reviews, as opposed to treating the variation as noise (e.g. exponential decay on
	time since publication), or ignoring the ground truth altogether \cite{2007liu-helpfulness}.

The most important of the three points I believe is the second one: identifying topics in 
	reviews which users find helpful; this is the topic I hope to pursue in my final project.


\bibliographystyle{acl}
\bibliography{proposal}
\end{document}
